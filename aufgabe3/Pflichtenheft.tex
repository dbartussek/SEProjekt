\title{
	FACHLICHE ANFORDERUNGEN\\
	Neuimplementierung von Donkey Kong
}
\author{
        Autoren: \\
	Dario Bartussek \\
	Ivan Lukic \\
	Majdi Fakher Aldein
}
\date{Stand: \today}

\documentclass[12pt]{article}

\usepackage{fancyhdr}
\usepackage{lipsum}
\usepackage{lastpage}
\usepackage{titleref}
\usepackage[ngerman]{babel}
\usepackage{datetime}
\usepackage{makecell}

\newdateformat{deformat}{\THEDAY.\THEMONTH.\THEYEAR}

\makeatletter
\newcommand*{\currentname}{\TR@currentTitle}
\makeatother

% Turn on fany style
\pagestyle{fancy}
% Clear the header and footer
\fancyhead{}
\fancyfoot{}

% Put the current section in the top right
\fancyhead[L]{\currentname}

\fancyfoot[L]{Stand: \deformat\today}

% Set the right side of the footer to be the page number
\fancyfoot[R]{Seite \thepage\ von \pageref{LastPage}}


\begin{document}
\maketitle

\newpage

\tableofcontents

\newpage


\section{Problemstellung}

\subsection{Motivation}
Das Retro-Donkey Kong soll erneurt werden, damit es vor allem für die jüngere Genaration attraktiver wird.
Somit soll einer der ersten Computerspiele zum neuen Leben gebracht werden.

\subsection{Ist-Zustand}
Man muss die Leitern hochklettern und bis zur höchsten Ebene gelangen um das Level zu schaffen. Es gibt insgesamt
4 Levels im Spiel, wobei jedes Level schwieriger als das vorherige ist. Die Hindernisse die Mario überwinden muss sind Fässer und Flammen.
Mario besitzt insgesamt 3 Leben. Falls man alle 3 Leben verliert folgt Game Over. Mario kann mit einem Hammer die Flammen und Fässer 
zerstören, wobei dann der Score steigt. Manche Leitern sind in der Mitte gebrochen, was bedeutet, das Mario nur bis zu einem gewissen Punkt
die Leiter erklimmen kann. Das Ziel des Spiels ist es die Prinzessin auf der obersten Ebene des letzten Levels zu retten.

\subsection{Ziel}
Die Steurung soll verbessert werden, damit nicht so schnell Frustration entsteht. 
Die Levels sollen aber trotzdem anspruchsvoll sein, damit der Spieler gefordert wird. Die Grafik soll auch ansprechender gestaltet werden.
Es sollen noch andere Hindernisse bzw. Gegner hinzugefügt werden, wobei manche Gegner levelspezifisch sind. 
Der Schwierigkeitsgrad der Gegner soll sich pro Level erhöhen.
Zusätzlich sollen neue Waffen und Items hinzugefügt werden.
Falls möglich soll ein Multiplayer-Modus entwickelt werden.

\subsection{Lösungsweg}
Das Spiel soll als flexible Java-Bibliothek geschrieben werden, die mit unterschiedlichen Forntends kombiniert werden kann. \\
Als Anwendung soll für diese ein User Interface entwickelt werden, vorzugsweise basierend auf JavaFX. \\
Der Multiplayer-Modus soll zunächst lokal, mittels unterschiedlicher Steuerungstasten implementiert werden.
Falls Zeit bleibt kann dieser Multiplayer über Sockets aufgebaut werden.

\subsection{Einschränkungen}
Das Spiel soll kein perfekter Nachbau des Originals werden, da dies zu kompliziert wäre und gewisse Aspekte verbessert werden sollen. \\
Für die Anwendung soll schlichte 2D Grafik verwendet werden. Aufwendigere Grafik würde den Scope des Projekts um ein vielfaches erhöhen.

\subsection{Verantwortliche}
Projektleiter ist Dario Bartussek.


\subsection{Stakeholder}
Die Entwicklung erfolgt im Auftrag von Prof. Dr. A. Metzner. \\
Tests und Codedokumentation werden primär vom Entwickler des jeweiligen Moduls angefertigt und durch Peer Review auf Fehler und fehlende Tests/Erklärungen überprüft. \\
Bugs werden vom Entwickler des jeweiligen Moduls in dem sie auftreten gelöst. \\

\newpage

\section{Anforderungen}

\subsection{Anforderungen an diese Version}
\subsubsection{Funktionale Anforderungen}

\newpage

\begin{center}

\begin{tabular}{ | c | c | }

\hline
Numer & 1 \\
\hline
Autor & \makecell{ Dario Bartussek \\
	Ivan Lukic \\
	Majdi Fakher Aldein } \\
\hline
Datum & 15.11.2019 \\
\hline
Titel & Spielfigur bewegen \\
\hline
Kurzbeschreibung &\makecell{ Der Spieler bewegt die Spielfigur mit den Pfeiltasten \\ nach Links, Rechts, Oben und Unten \\ und kann mit der Leertaste springen } \\
\hline
Primärer Akteur & Spieler \\
\hline
(Sekundärer Akteur) & \\
\hline
Vorbedingungen & Spiel muss laufen \\
\hline
Nachbedingung im Erfolgsfall & Marios Position wird geändert \\
\hline
Interaktionsfolge & 
	\begin{tabular}{ |c|c| }
	\hline
	Akteur & System \\
	\hline
	Spieler drückt Taste & \makecell{ System versucht \\ den anderen Punkt anzuzeigen } \\
	\hline
	& \makecell{ Mario an einem \\ neuen Punkt anzeigen } \\
	\hline
	\end{tabular} \\
\hline
\makecell{ Ausnahmen und Fehlerfälle \\ (und Extensions) } & 
	\makecell{ Mario kann sich nicht nach Rechts bewegen, \\ wenn er am rechten Rand ist. \\
	Mario kann sich nicht nach Links bewegen, \\ wenn er am linken Rand ist. \\
	Wenn Mario nicht an einer Leiter ist, \\ hat die Up-Taste keinen Effekt. \\
	Wenn Mario nicht an einer Leiter ist, \\ hat die Down-Taste keinen Effekt. }\\
\hline

\end{tabular}


\begin{tabular}{ | c | c | }

\hline
Numer & 3 \\
\hline
Autor & \makecell{ Dario Bartussek \\
	Ivan Lukic \\
	Majdi Fakher Aldein } \\
\hline
Datum & 15.11.2019 \\
\hline
Titel & Menüoption auswählen \\
\hline
Kurzbeschreibung &\makecell{ Neues spiel Starten oder weiter spielen \\
	High score / Credits anschauen } \\
\hline
Primärer Akteur & Spieler \\
\hline
(Sekundärer Akteur) & \\
\hline
Vorbedingungen & Spieler muss im Menü sein \\
\hline
Nachbedingung im Erfolgsfall & Teilgebiet des Menü wird geöffnet \\
\hline
Interaktionsfolge & 
	\begin{tabular}{ |c|c| }
	\hline
	Akteur & System \\
	\hline
	\makecell{ Spieler wählt mit \\ Up/Down Taste \\ eine Option aus } & \makecell{ Neue Option wird markiert } \\
	\hline
	\makecell{ Spieler muss mit \\ Enter bestätigen } & \\
	\hline
	\end{tabular} \\
\hline
\makecell{ Ausnahmen und Fehlerfälle \\ (und Extensions) } & \\
\hline

\end{tabular}


\begin{tabular}{ | c | c | }

\hline
Numer & 2 \\
\hline
Autor & \makecell{ Dario Bartussek \\
	Ivan Lukic \\
	Majdi Fakher Aldein } \\
\hline
Datum & 15.11.2019 \\
\hline
Titel & Fass zerschlagen \\
\hline
Kurzbeschreibung &\makecell{ Der Spieler kann mit dem Hammer Fässer zerstören. } \\
\hline
Primärer Akteur & Spieler \\
\hline
(Sekundärer Akteur) & \\
\hline
Vorbedingungen & \makecell{ Der Spieler muss den Hammer haben. } \\
\hline
Nachbedingung im Erfolgsfall & \\
\hline
Interaktionsfolge & 
	\begin{tabular}{ |c|c| }
	\hline
	Akteur & System \\
	\hline
	\makecell{ Spieler drückt Shift Taste } & \makecell{ Falls ein Fass getroffen wird, \\ wird es zerstört } \\
	\hline
	 & \makecell{ Scroe wird erhöht } \\
	\hline
	\end{tabular} \\
\hline
\makecell{ Ausnahmen und Fehlerfälle \\ (und Extensions) } & \\
\hline

\end{tabular}


\begin{tabular}{ | c | c | }

\hline
Numer & 5 \\
\hline
Autor & \makecell{ Dario Bartussek \\
	Ivan Lukic \\
	Majdi Fakher Aldein } \\
\hline
Datum & 15.11.2019 \\
\hline
Titel & Hammer berühren \\
\hline
Kurzbeschreibung &\makecell{ Der Spieler kann Hämmer aufheben. } \\
\hline
Primärer Akteur & Spieler \\
\hline
(Sekundärer Akteur) & \\
\hline
Vorbedingungen & \\
\hline
Nachbedingung im Erfolgsfall & \makecell{ Der Spieler hat einen Hammer. } \\
\hline
Interaktionsfolge & 
	\begin{tabular}{ |c|c| }
	\hline
	Akteur & System \\
	\hline
	\makecell{ Spieler berührt \\ den Hammer \\ mit seiner Figur } & \makecell{ System gibt der Spielfigur \\ den Hammer und startet \\ einen 30 Sekunden Timer. } \\
	\hline
	\end{tabular} \\
\hline
\makecell{ Ausnahmen und Fehlerfälle \\ (und Extensions) } & \\
\hline

\end{tabular}


\begin{tabular}{ | c | c | }

\hline
Numer & 4 \\
\hline
Autor & \makecell{ Dario Bartussek \\
	Ivan Lukic \\
	Majdi Fakher Aldein } \\
\hline
Datum & 15.11.2019 \\
\hline
Titel & Leiter hochklettern \\
\hline
Kurzbeschreibung &\makecell{ Der Spieler kann an Leitern nach oben klettern. } \\
\hline
Primärer Akteur & Spieler \\
\hline
(Sekundärer Akteur) & \\
\hline
Vorbedingungen & Mario muss unter einer Leiter stehen. \\
\hline
Nachbedingung im Erfolgsfall & \makecell{ Mario wird auf die höhere Ebene versetzt. } \\
\hline
Interaktionsfolge & 
	\begin{tabular}{ |c|c| }
	\hline
	Akteur & System \\
	\hline
	\makecell{ Spieler drückt die \\ Up Taste. } & \makecell{ System prüft, ob Mario \\ unter einer Leiter steht. } \\
	\hline
	& \makecell{ Mario wird auf die \\ höhere Ebene gesetzt. } \\
	\hline
	\end{tabular} \\
\hline
\makecell{ Ausnahmen und Fehlerfälle \\ (und Extensions) } & \\
\hline

\end{tabular}

\end{center}

\subsubsection{Anforderungen an Schnittstellen}
\subsubsection{Anforderungen an die Oberfläche}
\subsubsection{Nicht-funktionale Anforderungen}
\subsubsection{Anforderungen bezüglich der Migration}

\subsection{Wunschanforderungen (optional)}

\subsection{Anforderungen an spätere Versionen (optional)}

\subsection{Abgelehnte Anforderungen}

\subsection{Offene Punkte}

\end{document}
