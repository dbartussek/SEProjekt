\title{
	FACHLICHE ANFORDERUNGEN\\
	Neuimplementierung von Donkey Kong
}
\author{
        Autoren: \\
	Dario Bartussek \\
	Îvan Lukic \\
	Majdi Fakher Aldein \\
	Audai Brik
}
\date{Stand: \today}

\documentclass[12pt]{article}

\usepackage{fancyhdr}
\usepackage{lipsum}
\usepackage{lastpage}
\usepackage{titleref}
\usepackage[ngerman]{babel}
\usepackage{datetime}

\newdateformat{deformat}{\THEDAY.\THEMONTH.\THEYEAR}

\makeatletter
\newcommand*{\currentname}{\TR@currentTitle}
\makeatother

% Turn on fany style
\pagestyle{fancy}
% Clear the header and footer
\fancyhead{}
\fancyfoot{}

% Put the current section in the top right
\fancyhead[L]{\currentname}

\fancyfoot[L]{Stand: \deformat\today}

% Set the right side of the footer to be the page number
\fancyfoot[R]{Seite \thepage\ von \pageref{LastPage}}


\begin{document}
\maketitle

\newpage

\tableofcontents

\newpage


\section{Problemstellung}

\subsection{Motivation}
Das Retro-Donkey Kong soll erneurt werden, damit es vor allem für die jüngere Genaration attraktiver wird.
Somit soll einer der ersten Computerspiele zum neuen Leben gebracht werden.

\subsection{Ist-Zustand}
Man muss die Leitern hochklettern und bis zur höchsten Ebene gelangen um das Level zu schaffen. Es gibt insgesamt
4 Levels im Spiel, wobei jedes Level schwieriger als das vorherige ist. Die Hindernisse die Mario überwinden muss sind Fässer und Flammen.
Mario besitzt insgesamt 3 Leben. Falls man alle 3 Leben verliert folgt Game Over. Mario kann mit einem Hammer die Flammen und Fässer 
zerstören, wobei dann der Score steigt. Manche Leitern sind in der Mitte gebrochen, was bedeutet, das Mario nur bis zu einem gewissen Punkt
die Leiter erklimmen kann. Das Ziel des Spiels ist es die Prinzessin auf der obersten Ebene des letzten Levels zu retten.

\subsection{Ziel}
Die Steurung soll verbessert werden, damit nicht so schnell Frustration entsteht. 
Die Levels sollen aber trotzdem anspruchsvoll sein, damit der Spieler gefordert wird. Die Grafik soll auch ansprechender gestaltet werden.
Es sollen noch andere Hindernisse bzw. Gegner hinzugefügt werden, wobei manche Gegner levelspezifisch sind. 
Der Schwierigkeitsgrad der Gegner soll sich pro Level erhöhen.
Zusätzlich sollen neue Waffen und Items hinzugefügt werden.
Falls möglich soll ein Multiplayer-Modus entwickelt werden.

\subsection{Lösungsweg}
Das Spiel soll als flexible Java-Bibliothek geschrieben werden, die mit unterschiedlichen Forntends kombiniert werden kann. \\
Als Anwendung soll für diese ein User Interface entwickelt werden, vorzugsweise basierend auf JavaFX. \\
Der Multiplayer-Modus soll zunächst lokal, mittels unterschiedlicher Steuerungstasten implementiert werden.
Falls Zeit bleibt kann dieser Multiplayer über Sockets aufgebaut werden.

\subsection{Einschränkungen}
Das Spiel soll kein perfekter Nachbau des Originals werden, da dies zu kompliziert wäre und gewisse Aspekte verbessert werden sollen. \\
Für die Anwendung soll schlichte 2D Grafik verwendet werden. Aufwendigere Grafik würde den Scope des Projekts um ein vielfaches erhöhen.

\subsection{Verantwortliche}
Projektleiter ist Dario Bartussek.


\subsection{Stakeholder}
Die Entwicklung erfolgt im Auftrag von Prof. Dr. A. Metzner. \\
Tests und Codedokumentation werden primär vom Entwickler des jeweiligen Moduls angefertigt und durch Peer Review auf Fehler und fehlende Tests/Erklärungen überprüft. \\
Bugs werden vom Entwickler des jeweiligen Moduls in dem sie auftreten gelöst. \\

\newpage

\section{Anforderungen}

\subsection{Anforderungen an diese Version}
\subsubsection{Funktionale Anforderungen}
\subsubsection{Anforderungen an Schnittstellen}
\subsubsection{Anforderungen an die Oberfläche}
\subsubsection{Nicht-funktionale Anforderungen}
\subsubsection{Anforderungen bezüglich der Migration}

\subsection{Wunschanforderungen (optional)}

\subsection{Anforderungen an spätere Versionen (optional)}

\subsection{Abgelehnte Anforderungen}

\subsection{Offene Punkte}

\end{document}
